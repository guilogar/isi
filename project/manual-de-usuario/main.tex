\documentclass[a4paper,12pt]{article}
\usepackage[utf8]{inputenc}
\usepackage[spanish]{babel}
\usepackage{graphicx}
\usepackage{tabularx}
\usepackage{float}
\usepackage{hyperref}
\usepackage{blindtext}
\usepackage{adjustbox}
\usepackage[toc,page]{appendix}
\usepackage{enumitem}
\setlist{  
  listparindent=\parindent,
  parsep=0pt,
}
\usepackage{listings}
\usepackage{tabularx}  % for 'tabularx' environment and 'X' column type
\usepackage{ragged2e}  % for '\RaggedRight' macro (allows hyphenation)
\newcolumntype{Y}{>{\RaggedRight\arraybackslash}X} 
\usepackage{booktabs}  % for \toprule, \midrule, and \bottomrule macros 
\usepackage[raggedright]{titlesec}
\renewcommand{\lstlistingname}{Código}
\usepackage[nottoc,notlot,notlof]{tocbibind}

\setcounter{secnumdepth}{4}
\setcounter{tocdepth}{2}


 %%%%%%%%%%%%%%% codigos %%%%%%%%%%%%%%% 
 \usepackage[T1,OT1]{fontenc}
\usepackage{color}
 
\definecolor{codegreen}{rgb}{0,0.6,0}
\definecolor{codegray}{rgb}{0.5,0.5,0.5}
\definecolor{codepurple}{rgb}{0.58,0,0.82}
\definecolor{backcolour}{rgb}{0.95,0.95,0.92}
 
\lstdefinestyle{mystyle}{
    backgroundcolor=\color{backcolour},   
    commentstyle=\color{codegreen},
    keywordstyle=\color{magenta},
    numberstyle=\tiny\color{codegray},
    stringstyle=\color{codepurple},
    basicstyle=\footnotesize,
    breakatwhitespace=false,         
    breaklines=true,                 
    captionpos=b,                    
    keepspaces=true,                 
    numbers=left,                    
    numbersep=5pt,                  
    showspaces=false,                
    showstringspaces=false,
    showtabs=false,                  
    tabsize=2
}

\lstdefinelanguage{oaw}{
  morekeywords={import, let, then, Void, extension, JAVA,
  IMPORT, DEFINE, ENDDEFINE, LET, ENDLET, FOR, FILE, ENDFILE, ITERATOR, FOREACH,
  AS, IF, ENDFOREACH, ENDIF, EXPAND, INSTANCEOF, USING, SEPARATOR, CSTART, CEND, 
  PROTECT, ENDPROTECT, ID, EXTENSION,   
  context, ERROR, WARNING, INFO, enum},
  morecomment=[s]{«REM»}{«ENDREM»},
  morecomment=[s]{«REM}{»},
  escapechar={@},
  % required for use with UTF-8
  literate={«}{\guillemotleft}{1}
           {»}{\guillemotright}{1}
}

\lstdefinelanguage{xml}{
  morekeywords={xml, workflow, bean, platformUri, registerEcoreFile, component, uri, modelSlot, metaModel, checkFile, property, emfAllChildrenSlot,directory, expand, outlet, postprocessor}
}

\lstdefinelanguage{emf}{
  morekeywords={gmf,@Ecore,@namespace,class,package,attr,enum,@gmf,node,compartment,@gmf . compartment}
}

\lstdefinelanguage{atl}{
  morekeywords={rule, from, to, helper, context, module, create, def}
}
 
\lstset{style=mystyle}
%%%%%%%%%%%%%%%%%%%%%%%%%%%%%%%%%%%%%%%%%%%%%%

\title{\textbf{
  Manual de Usuario e Instalación\\SmartRural
} \\ \small Ingeniería de Sistemas de la Información}
\author{Guillermo López García}
\date{\today}


\begin{document}

\maketitle

\begin{figure}[ht]
    \centering
	\includegraphics[scale=0.1]{uca.png}
\end{figure}

\clearpage

\tableofcontents
\clearpage

\section{SDKman}
Uno de los sistemas a instalar es el SDKman.

Esta herramienta nos permitira instalar distintos SDK para el funcionamiento de la aplicación encargada de generar
los datos aleatorios. Así pues, seguimos los pasos de \url{https://sdkman.io/install} y posteriormente pasamos
a instalar las herramientas necesarias.

\subsection{Java 8}
Una vez tengamos SDKman instalado, instalar java es tán fácil como escribir: 

sdk install java 8.0.265-open

\subsection{Maven}
Una vez tengamos instalado java, también necesitaremos maven. Para ello, escribimos en la consola de comandos:

sdk install maven

\section{Docker}
En esta sección, aprenderemos a instalar docker.

Este sistema de virtualización nos permitira virtualizar la base de datos y no tener que depender de una instalación
en local obsoleta y en muchas ocasiones, imposible de hacer con un versión actual de un sistema operativo moderno como
Windows 10, Linux en cualquier de sus variantes, Mac OSX, etc.

Así pues, con seguir las instrucciones escritas en el siguiente enlace,
\url{https://docs.docker.com/engine/install/},
tendríamos instalado docker en el sistema que queremos.

\subsection{docker-compose}
Pero docker solo no nos sirve, nos sirve una variante o herramienta del mismo llamado docker-compose, que sirve
para generar contenedores y servicios de forma fácil y sencilla con un simple fichero.

Así pues, seguimos las instrucciones siguientes \url{https://docs.docker.com/compose/install/}, tendriamos el
docker-compose instalado en nuestro sistema.

\subsection{MySQL}
Por último, para levantar el servicio MySQL, escribiremos el comando docker-compose up -d, en el mismo directorio
que donde se encuentre el fichero \textbf{docker-compose.yml} que tiene el siguiente contenido.

\begin{lstlisting}[language=xml,caption=docker-compose.yml]
  version: '3.3'
  services:
    smartrural:
      image: mysql:5.7
      restart: always
      environment:
        MYSQL_DATABASE: 'smartrural'
        MYSQL_USER: isi
        MYSQL_PASSWORD: 'isi'
        MYSQL_ROOT_PASSWORD: 'root'
      ports:
        - '3306:3306'
      expose:
        - '3306'
      volumes:
        - smartrural:/var/lib/mysql
  volumes:
    smartrural:
\end{lstlisting}

Además, si se quiere usar datos de ejemplo, se podría importar el fichero smartrural dump
con extensión sql con el siguiente comando:

\textbf{mysql -h 127.0.0.1 -u root -P 3306 -p smartrural < smartrural dump.sql}

\section{NVM}
Por otra parte, en la instalación de NodeJS, para seguir la misma dinámica que anteriormente, vamos
a usar un gestor de instalación de NodeJs. En concreto, vamos a usar el Node Version Manager.

Para instalarlo, no hay más que seguir los pasos descritos en el
siguiente enlace: \url{https://github.com/nvm-sh/nvm}

\subsection{NodeJS}
Una vez tengamos instalado correctamente NVM, instalar NodeJS es tán fácil como escribir
nvm install 14

Notesé que hemos instalado la versión 14, pero podríamos instalar cualquiera anterior o posterior.

\subsection{NPM}
NPM se instala automáticamente al instalar la versión de NodeJS, con lo cual, no hay que hacer nada extra.

\section{Ionic}
Para instalar Ionic en su última versión, la cuál permite crear y ejecutar proyectos con React.JS, 
se debe ejecutar el siguiente comando: \textbf{npm install -g @ionic/cli}

Para saber si esta correctamente instalado, ejecutar posteriormente \textbf{ionic --version}

\section{Anypoint Studio}
Para instalar Anypoint Studio, es tán sencillo como ir a la siguiente url
\url{https://www.mulesoft.com/lp/dl/studio/previous}, seleccionar 
\textbf{Mule 3.9 and Studio 6.6}, seleccionar tu sistema operativo y darle a descargar.

Una vez lo tengamos instalado, lo ejecutamos y listo. Gracias a la versión 8 de Java instalada antes,
no deberiamos tener ningún problema.

\section{Ejecución del Proyecto}
Aquí, mostraremos los pasos a seguir para ejecutar el proyecto de forma correcta:

\begin{itemize}
  \item Generar datos: java -jar AutomationSimulatorRefactor-1.0-SNAPSHOT-jar-with-dependencies.jar.
  \item MySQL: levantamos la base de datos con docker-compose up -d.
  \item Anypoint Studio: abrimos el proyecto con el Anypoint Studio y le damos a ejecutar.
  \item Backend: nos movemos al directorio del backend y escribimos \textbf{node .}
  \item Frontend: nos movemos al directorio del frontend y escribimos \textbf{ionic serve}
  \item Navegador: abrimos el navegador en la dirección http://localhost:8100 y listo para usar el entorno.
\end{itemize}

Por último, si queremos que el proyecto en Anypoint Studio funcione con nuestras credenciales, deberemos ir
al archivo \textbf{smart-rural/src/main/app/mule-app.properties} y poner los valores que queramos a las credenciales
del email y el path de los directorios de nuestros eventos.

Mostramos a continuación el contenido del mismo:

\begin{lstlisting}[language=xml,caption=mule-app.properties]
    #** GENERATED CONTENT ** Mule Custom Properties file.
    #Wed Nov 11 10:04:08 CET 2020
    email.username=xxxxxxxxxxx@gmail.com
    email.from=xxxxxxxxxxxxxxx@gmail.com
    email.password=xxxxxxxxxxxxxxxxxx
    email.to=xxxxxxxxxxxxxxxx@gmail.com
    patterns.path=/home/guilogar
\end{lstlisting}

Claro esta, el valor de las credenciales de email ha sido cambiado para proteger nuestra privacidad.

\clearpage

\nocite{*}
\bibliographystyle{IEEEtran}
\bibliography{bib}

\end{document}
